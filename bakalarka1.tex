%% LyX 1.5.5 created this file.  For more info, see http://www.lyx.org/.
%% Do not edit unless you really know what you are doing.
\documentclass[a4paper,twoside,czech,czech,openright,cleardoubleempty,BCOR10mm,DIV11]{scrreprt}
\usepackage[T1]{fontenc}
\usepackage[utf8]{inputenc}
\usepackage{array}
\usepackage{longtable}
\usepackage{varioref}
\usepackage{wrapfig}
\usepackage{fancybox}
\usepackage{calc}
\usepackage{framed}
\usepackage{url}
\usepackage{graphicx}

\makeatletter

%%%%%%%%%%%%%%%%%%%%%%%%%%%%%% LyX specific LaTeX commands.
\providecommand{\LyX}{L\kern-.1667em\lower.25em\hbox{Y}\kern-.125emX\@}
\newcommand{\lyxline}[1][1pt]{%
  \par\noindent%
  \rule[.5ex]{\linewidth}{#1}\par}
\newcommand{\noun}[1]{\textsc{#1}}
%% Special footnote code from the package 'stblftnt.sty'
%% Author: Robin Fairbairns -- Last revised Dec 13 1996
\let\SF@@footnote\footnote
\def\footnote{\ifx\protect\@typeset@protect
    \expandafter\SF@@footnote
  \else
    \expandafter\SF@gobble@opt
  \fi
}
\expandafter\def\csname SF@gobble@opt \endcsname{\@ifnextchar[%]
  \SF@gobble@twobracket
  \@gobble
}
\edef\SF@gobble@opt{\noexpand\protect
  \expandafter\noexpand\csname SF@gobble@opt \endcsname}
\def\SF@gobble@twobracket[#1]#2{}
%% Because html converters don't know tabularnewline
\providecommand{\tabularnewline}{\\}

%%%%%%%%%%%%%%%%%%%%%%%%%%%%%% Textclass specific LaTeX commands.
\newenvironment{lyxcode}
{\begin{list}{}{
\setlength{\rightmargin}{\leftmargin}
\setlength{\listparindent}{0pt}% needed for AMS classes
\raggedright
\setlength{\itemsep}{0pt}
\setlength{\parsep}{0pt}
\normalfont\ttfamily}%
 \item[]}
{\end{list}}

%%%%%%%%%%%%%%%%%%%%%%%%%%%%%% User specified LaTeX commands.
%<-------------------------------společná nastavení------------------------------>
\usepackage[czech]{babel}%počeštění názvů (Obsah, Kapitola, Literatura atp.)
\usepackage[]{hyperref} %odkazy v  pdf jsou klikací s barevnými rámečky
\usepackage[numbers,sort&compress]{natbib} %balíček pro citace literatury  
\usepackage{hypernat}%interakce mezi hyperref a natbib
\newcommand{\BibTeX}{{\sc Bib}\TeX}%BibTeX logo
\hypersetup{   % Nastavení polí PDF dokumentu 
pdftitle={Sablona pro psani zaverecnych praci v LyXu},%   
pdfauthor={Vitezslav Vydra},%  
pdfsubject={},%   
pdfkeywords={\v{s}ablona,LaTeX,LyX}%                             
}
\usepackage{multicol}




%<-----------------------------volání stylů----------------------------------------->
% (znak % je označení komentáře: co je za ním, není aktivní)
%<------------------------------------písmo----------------------------------------->
%\usepackage{packages/bc-latinmodern}
%\usepackage{packages/bc-times}
\usepackage{packages/bc-palatino}
%\usepackage{packages/bc-iwona}
%\usepackage{packages/bc-helvetika}


%<------------------------------záhlaví stránek------------------------------------>
%\usepackage{packages/bc-headings}
\usepackage{packages/bc-fancyhdr}

%<------------------------------hlavičky kapitol------------------------------------>
%\usepackage{packages/bc-neueskapitel}
\usepackage{packages/bc-fancychap}

\makeatother

\usepackage{babel}

\begin{document}
~\thispagestyle{empty}{\small ~\vfill{}
}{\small \par}

\noindent {\small Na tomto místě mohou být napsána případná poděkování
(vedoucímu práce, konzultantovi, tomu kdo půjčil software, literaturu,
poskytl data apod.). \newpage{}}{\small \par}

~\thispagestyle{empty}\vfill{}
Tato stránka je tzv. protititul a je graficky součástí titulní stránky.
Nechte ji prázdnou, nebo na ni umístěte vhodnou fotografii či ilustraci.

\cleardoublepage{}~\thispagestyle{empty}\begin{center}\pagenumbering{roman}\vspace{10mm}


\textsf{\textsc{\noun{\LARGE České vysoké učení technické v Praze}}}\\
\vspace{0.5em}
\textsf{\textsc{\noun{\LARGE Fakulta stavební}}}\\
\vspace*{1em}
\textsf{\textsc{\noun{\Large katedra fyziky}}}\vspace{15mm}


%%% Aby vložení loga  správně fungovalo, je třeba mít soubor lev.png nahraný v pracovním adresáři,
%%% tj. v adresáři, kde se nachází překládaný zdrojový soubor. 
\includegraphics[width=0.3\textwidth]{obrazky/lev}\vspace{15mm}


\textsf{\huge BAKALÁŘSKÁ/DIPLOMOVÁ PRÁCE}{\huge \par}

\vspace{15mm}


\textsf{\LARGE Název práce}{\LARGE \par}

\vspace{10mm}


\end{center} 

\vspace*{\fill}


\vspace{10mm}


\begin{description}
\item [{{\large Autor:}}] \noindent \textsf{\large Jméno autora}{\large \par}
\item [{{\large Vedoucí~práce:}}] \noindent \textsf{\large Doc. RNDr.
Vítězslav Vydra, CSc.}{\large \hfill{}}\textsf{\large Praha, 2008}{\large{}
% doplňte rok vzniku vaší bakalářské práce
}{\large \par}
\end{description}
\clearpage{}

{\small \thispagestyle{plain}\addcontentsline{toc}{chapter}{Abstrakt} }{\small \par}

\noindent {\small ~\vfill{}
}{\small \par}

\begin{description}
\item [{{\small Název~práce:}}] \noindent {\small Název bakalářské práce}{\small \par}
\item [{{\small Autor:}}] \noindent {\small Jméno autora}{\small \par}
\item [{{\small Katedra~(ústav):}}] \noindent Kate{\small dra fyziky}{\small \par}
\item [{{\small Vedoucí~bakalářské~práce:}}] \noindent Doc. RNDr. Vítězslav
Vydra, CSc.
\item [{{\small e-mail~vedoucího:}}] \noindent {\small vydra@fsv.cvut.cz}\\
{\small \par}
\item [{{\small Abstrakt}}] \noindent {\small V předložené práci studujeme...
Uvede se abstrakt v rozsahu 80 až 200 slov. Lorem ipsum dolor sit
amet, consectetuer adipiscing elit. Ut sit amet sem. Mauris nec turpis
ac sem mollis pretium. Suspendisse neque massa, suscipit id, dictum
in, porta at, quam. Nunc suscipit, pede vel elementum pretium, nisl
urna sodales velit, sit amet auctor elit quam id tellus. Nullam sollicitudin.}{\small \par}
\item [{{\small Klíčová~slova:}}] \noindent {\small klíčová slova (3 až
5)}\\
{\small \lyxline{\small}}{\small \par}
\item [{{\small Title:}}] \noindent {\small Název bakalářské práce v angličtině}{\small \par}
\item [{{\small Author:}}] \noindent {\small Jméno autora}{\small \par}
\item [{{\small Department:}}] \noindent {\small Název katedry či ústavu
v angličtině}{\small \par}
\item [{{\small Supervisor:}}] \noindent {\small Jméno s tituly jako v
české verzi, event. pracoviště}{\small \par}
\item [{{\small Supervisor's~e-mail~address:}}] \noindent {\small e-mailová
adresa vedoucího}\\
{\small \par}
\item [{{\small Abstract}}] \noindent {\small In the present work we study
... Uvede se anglický abstrakt v rozsahu 80 až 200 slov. Lorem ipsum
dolor sit amet, consectetuer adipiscing elit. Ut sit amet sem. Mauris
nec turpis ac sem mollis pretium. Suspendisse neque massa, suscipit
id, dictum in, porta at, quam. Nunc suscipit, pede vel elementum pretium,
nisl urna sodales velit, sit amet auctor elit quam id tellus. Nullam
sollicitudin. Donec hendrerit. Aliquam ac nibh. Vivamus mi. Sed felis.
Proin pretium elit in neque. Pellentesque at turpis. Maecenas convallis.
Vestibulum id lectus. }{\small \par}
\item [{{\small Keywords:}}] \noindent {\small klíčová slova (3 až 5) v
angličtině}{\small \par}
\end{description}
\cleardoublepage{}\thispagestyle{empty}~{\small \addcontentsline{toc}{chapter}{Zadání
práce} }{\small \par}



\newpage{}\thispagestyle{empty}~



\newpage{}\thispagestyle{plain}

{\small %\setcounter{page}{3} % nastavení číslování stránek
\ }{\small \par}

\noindent {\small \vfill{}
 % nastavuje dynamické umístění následujícího textu do spodní části stránky
~}{\small \par}

\noindent {\small Prohlašuji, že jsem svou bakalářskou práci napsal(a)
samostatně a výhradně s použitím citovaných pramenů. Souhlasím se
zapůjčováním práce a jejím zveřejňováním.}{\small \par}

{\small \bigskip{}
}\noindent {\small{} V Praze dne \today\hspace{\fill}Jméno Příjmení
+ podpis}\\
{\small{} % doplňte patřičné datum, jméno a příjmení
}{\small \par}

{\small %%%   Výtisk pak na tomto míste nezapomeňte PODEPSAT!
%%%                                         *********
}{\small \par}

\cleardoublepage{}\thispagestyle{empty}{\small \tableofcontents{}% vkládá automaticky generovaný obsah dokumentu
\cleardoublepage{}}{\small \par}

\pagenumbering{arabic}%start arabic pagination from 1 


\chapter{Úvod}

V současné době existuje nespočet frameworků\footnote{jeho cílem je převzetí typických problémů dané oblasti, čímž se usnadní vývoj tak, aby se návrháři a vývojáři mohli soustředit pouze na své zadání}) pro vývoj webových aplikací ve spoustě programovacích jazycích. 
Výběr takového nástroje pak může být pro danou aplikaci klíčový. Vzhledem k faktu, že je s aplikací po celý životní cyklus, může se s časem stát svazujícím a nedostačujícím. Na reimplementaci však již není čas nebo peníze. Většina webových aplikací tak dříve nebo později narazí na na problematiku škálování, kdy je třeba rozložit aplikaci na více serverů ať už pro zajištění vysoké dostupnosti nebo kvůli velké výpočetní náročnosti. Dnes také není nic neobvyklého, že aplikaci najednou začnou navštěvovat tisíce klientů za minutu. Z dříve rychlé a stabilní aplikace se tak může stát často padající aplikace s nepřiměřenou odezvou.

Právě proto, jsem se rozhodl pro hlubší zkoumání v dané oblasti webových aplikací. V první části bakalářské práce popisuji architekturu a jednotlivé technologie, které mě motivovaly k hlubšímu studiu platformy Vert.x. V hlavní části práce následuje návrh, implementace a nasazení jednostránkové aplikace. V závěru je pak shrnutí kladů a záporů platformy.

\section{Cíl a metodika práce}

Hlavním cílem mé práce je zjištění zda-li se platforma Vert.x hodí pro vývoj distribuovaných webových aplikací. 
Vytvoření jednoduchého webového editoru pro správu myšlenkových map. %Jednostránkové webové aplikace pro kolaborativní práci s mindmapami. % 
Na této jednoduché aplikaci bude demonstrována architektura a nasazení aplikace na více serverů pro zajištění vysoké dostupnosti.
%proces vývoje a nasazení webové aplikace pod platformou Vert.x. Vzhledem k rozsahu práce budou popsány spíše principy a architektura daného řešení než implementační detaily. 
Zdrojové kódy včetně návodu na spuštění aplikace jsou umístěny veřejně na serveru Github\footnote{www.github.com/michaelkuty} a na přiloženém médiu.

Je nutné uchopit problematiku platformy Vert.x v širších souvislostech, proto se v práci snažím neopomenout všechny technologie, které s Vert.x souvisí, z kterých Vert.x vychází nebo které přímo integruje. V teoretické části bude čtenář seznámen s důležitými filozofiemi, které platforma nabízí. %A to jak událostmi řízenou architekturou, kterou platforma převzala z dnes již dobře známého frameworku Node.js\footnote{Serverový framework, postavený na modelu událostmi řízeného programování}. Tak především polyglot programování s jednoduchým konkurenčním modelem a možností sdílet data mezi jednotlivými vlákny bez nutnosti zámků.

Cílem teoretické části je tedy popsat jednotlivé architektonické prvky a komponenty platformy, jejich účel či problém, který řeší. V závěru teoretické části bude platforma srovnána s nástrojem Node.js. Srovnání bude obsahovat test výkonnosti a porovnání vlastností.

V praktické části bude vytvořen editor pro správu a tvorbu myšlenkových map. Tyto mapy bude moci spravovat více uživatelů najednou v reálném čase. Budou popsány a vysvětleny aspekty komunikace v reálném čase včetně samotného nasazení webové aplikace na jednotlivé servery, kde bude prověřena funkčnost distribuovaného provozu aplikace v režimu vysoké dostupnosti.

\section{Postup a předpoklady práce}

Práce předpokládá základní znalost programovacího jazyku Java a JavaScript. Teoretická část se neomezuje pouze na nezbytný popis technologií potřebných k realizaci malé jednostránkové webové aplikace. Představuje stručný pohled na celou platformu Vert.x. Teoretická část může být použita jako odraz k hlubšímu studiu daných technologií a konceptů. Praktická část bude prokládáná ukázkami kódu nebo příkazy souvisejícími s vývojem webových aplikací. Práce předpokládá znalost základní terminologie související s programováním obecně. Méně zažité pojmy budou vysvětleny poznámkou pod čarou.

Při vývoji webové aplikace budou použity následující softwarové technologie:
\begin{itemize}
\item Vert.x 2.1.2: platforma pro vývoj webových aplikací
\item MongoDB: dokumentově orientovaná NoSQL\footnote{databázový koncept, ve kterém datové úložiště i zpracování dat používají jiné prostředky než tabulková schémata tradiční relační databáze} databáze
\item D3.js: framework pro práci s grafy
\item JQuery framework pro práci s GUI(Graphical user interface)
\end{itemize}

\include{sablona}

\include{uvod_LyX}


\chapter[Závěr ]{Dobrá rada na závěr}

\LyX{} je vynikající editor, který vám usnadní napsání rozsáhlejší
práce typu bakalářka nebo diplomka. Editor si hravě poradí s komplikovanými
úlohami jako je vkládání křížových odkazů, vytvoření seznamu literatury
a citování literatury v textu, vytvoření obsahu a rejstříku. Bez většího
úsilí bude vaše práce  typograficky na úrovni.

Používáte-li %
\marginpar{\textbf{\Huge !}%
}\LyX{} jen na psaní bakalářky, \emph{nesnažte se} naučit vše, co umí!
Zabralo by to více času než celá bakalářka! Naučte se jen pár nezbytností
a pište a pište a pište! Až budete mít dopsán a zkontrolován text,
můžete si pohrát s výběrem vzhledu vhodného pro vaši práci, s výběrem
písma, typu záhlaví stránek, hlaviček kapitol atd. Teprve nakonec
udělejte závěrečnou typografickou revizi textu. Zejména zkontrolujte
polohu plovoucích objektů (případně je přemístěte na vhodnější místo)
a odstraňte vdovy a sirotky (osamělé řádky)%
\footnote{Nejsnáze odstranit tak, že z textu vypustíte (nebo do něj přidáte)
pár slov či vět anebo úpravou odstavců.%
}.


\begin{thebibliography}{10}
\bibitem{Thesis-templates}\emph{\LyX{} and \LaTeX{} Thesis Themplates}
{[}online]. {[}cit. 2008-09-28]. Dostupný z WWW: \url{http://www.thesis-template.com/}

\bibitem{Diplomka-v-LaTeXu}Pele,\emph{ Diplomka v \LaTeX{}u} {[}online].
{[}cit. 2008-09-28]. Dostupný z WWW: \url{pele.gzk.cz/node/37}

\bibitem{Jirkovy-stranky}Roubal, Jiří\emph{. Jirkovy stránky}~{[}online].
{[}cit. 2008-09-28]. Dostupný z WWW: \url{dce.felk.cvut.cz/roubal/}

\bibitem{Lyx-cesky}Vydra, Vítězslav.\emph{ Počeštění \LyX{}u}~{[}online].
2008 {[}cit. 2008-09-28]. Dostupný z WWW: \url{people.fsv.cvut.cz/~vydra/lyxcesky.htm}

\bibitem{Tex-Gyre}\emph{Písmo \TeX-Gyre}~{[}online]. {[}cit. 2008-09-28].
Dostupný z WWW: \url{www.gust.org.pl/projects/e-foundry/tex-gyre}

\bibitem{NeuesKapitel}\emph{Neues Kapitel-Layout}~{[}online]. {[}cit.
2008-09-28]. Dostupný z WWW: \url{www.thesis-template.de/archives/5#more-5}

\bibitem{Vavreckova}Vavrečková, Šárka. \emph{Úprava dokumentů}~{[}online].
{[}cit. 2008-09-28]. Dostupný z WWW:\url{axpsu.fpf.slu.cz/~vav10ui/obsahy/dipl/typografie.pdf}

\bibitem{diplPraceSLU}\emph{Zdroje informací pro diplomové práce,
SLU}~{[}online]. {[}cit. 2008-09-28]. Dostupný z WWW: \url{axpsu.fpf.slu.cz/~vav10ui/obsahy/dipl/typodipl.html}

\bibitem{V-cem}\emph{V čem napsat diplomovou práci}~{[}online].
{[}cit. 2008-09-28]. Dostupný z WWW: \url{www.student.cvut.cz/cwut/index.php/Diplomová_práce#V_.C4.8Dem_napsat_diplomovou_pr.C3.A1ci}

\bibitem{Menousek}Menoušek, Jiří. \emph{Jak (ne)napsat diplomovou
a dizertační práci}~{[}online]. {[}cit. 2008-09-28]. Dostupný z WWW:
\url{www.csmo.cz/other/dizert.php}

\bibitem{Polach}Polách, Eduard. \emph{Pravidla sazby diplomových
prací}~{[}online]. {[}cit. 2008-09-28]. Dostupný z WWW: \url{home.pf.jcu.cz/~edpo/pravidla/pravidla.html}

\bibitem{Citace}Farkašová, Blanka, Krčál, Martin. \emph{Projekt bibliografické
citace}~{[}online]. {[}cit. 2008-09-28]. Dostupný z WWW: \url{www.citace.com}.

\end{thebibliography}
~\\
Tento seznam literatury byl vytvořen přímo v Lyxu pomocí stylu ,,Bibliografie{}``
a generátoru citací~\cite{Citace}. Pořadí citací je takové, jak
je sami napíšeme.

\bibliographystyle{csplainnat}
\bibliography{bakalarka}
~\\
~\\
Tento seznam literatury byl vytvořen pomocí \BibTeX u s použitím
stylu \texttt{csplainnat}. Citace jsou automaticky seřazeny podle
abecedy.

\addcontentsline{toc}{chapter}{Literatura} 

\cleardoublepage{}

\appendix
\pagenumbering{Roman}\addcontentsline{toc}{part}{Přílohy}\thispagestyle{empty}  \renewcommand{\appendixname}{P\v{r}iloha}%%přílohy, číslování římskými


\part*{Přílohy}

\listoffigures

\listoftables

\listoflistings


\end{document}
