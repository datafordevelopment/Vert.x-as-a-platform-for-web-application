\pagenumbering{arabic}%start arabic pagination from 1 


\chapter{Úvod}

V současné době existuje nespočet frameworků\footnote{jeho cílem je převzetí typických problémů dané oblasti, čímž se usnadní vývoj tak, aby se návrháři a vývojáři mohli soustředit pouze na své zadání}) pro vývoj webových aplikací ve spoustě programovacích jazycích. 
Výběr takového nástroje pak může být pro danou aplikaci klíčový. Vzhledem k faktu, že je s aplikací po celý životní cyklus, může se s časem stát svazujícím a nedostačujícím. Na reimplementaci však již není čas nebo peníze. Většina webových aplikací tak dříve nebo později narazí na na problematiku škálování, kdy je třeba rozložit aplikaci na více serverů ať už pro zajištění vysoké dostupnosti nebo kvůli velké výpočetní náročnosti. Dnes také není nic neobvyklého, že aplikaci najednou začnou navštěvovat tisíce klientů za minutu. Z rychlé a stabilní aplikace se tak může stát často padající aplikace s nepřiměřenou odezvou.

Právě proto, jsem se rozhodl k hlubšímu zkoumání v dané oblasti webových aplikací. V první části bakalářské práce je popsána architektura a jednotlivé technologie, které mě motivovaly k hlubšímu studiu platformy Vert.x. V hlavní části práce následuje návrh, implementace a nasazení jednostránkové aplikace. V závěru je pak shrnutí kladů a záporů platformy.

\section{Cíl a metodika práce}

Hlavním cílem práce bude zjištění zda-li se platforma Vert.x hodí pro vývoj distribuovaných webových aplikací. 
Vytvoření jednoduchého webového editoru pro správu myšlenkových map. %Jednostránkové webové aplikace pro kolaborativní práci s mindmapami. % 
Na této jednoduché aplikaci bude demonstrována architektura a nasazení aplikace na více serverů pro zajištění vysoké dostupnosti.
%proces vývoje a nasazení webové aplikace pod platformou Vert.x. Vzhledem k rozsahu práce budou popsány spíše principy a architektura daného řešení než implementační detaily. 
Zdrojové kódy včetně návodu na spuštění aplikace jsou umístěny veřejně na serveru Github\footnote{www.github.com/michaelkuty} a na přiloženém médiu.

Je nutné uchopit problematiku platformy Vert.x v širších souvislostech, proto se práce snaží neopomenout všechny technologie, které s Vert.x souvisí, z kterých Vert.x vychází nebo které přímo integruje. V teoretické části bude čtenář seznámen s důležitými filozofiemi, které platforma nabízí. %A to jak událostmi řízenou architekturou, kterou platforma převzala z dnes již dobře známého frameworku Node.js\footnote{Serverový framework, postavený na modelu událostmi řízeného programování}. Tak především polyglot programování s jednoduchým konkurenčním modelem a možností sdílet data mezi jednotlivými vlákny bez nutnosti zámků.

Cílem teoretické části je tedy popsat jednotlivé architektonické prvky a komponenty platformy, jejich účel či problém, který řeší. V závěru teoretické části bude platforma srovnána s nástrojem Node.js. Srovnání bude obsahovat test výkonnosti a porovnání vlastností.

V praktické části bude vytvořen editor pro správu a tvorbu myšlenkových map. Tyto mapy bude moci upravovat více uživatelů najednou v reálném čase. Budou popsány a vysvětleny aspekty komunikace v reálném čase včetně samotného nasazení webové aplikace na jednotlivé servery, kde bude prověřena funkčnost distribuovaného provozu aplikace v režimu vysoké dostupnosti.

\section{Postup a předpoklady práce}

Práce předpokládá základní znalost programovacího jazyku Java a JavaScript. Teoretická část se neomezuje pouze na nezbytný popis technologií potřebných k realizaci malé jednostránkové webové aplikace. Představuje stručný pohled na celou platformu Vert.x. Teoretická část může být použita jako odraz k hlubšímu studiu daných technologií a konceptů. Praktická část bude prokládáná ukázkami kódu nebo příkazy souvisejícími s vývojem webových aplikací. Práce předpokládá znalost základní terminologie související s programováním obecně. Méně zažité pojmy budou vysvětleny poznámkou pod čarou.

Při vývoji webové aplikace budou použity následující softwarové technologie:
\begin{itemize}
\item Vert.x 2.1.2: platforma pro vývoj webových aplikací
\item MongoDB: dokumentově orientovaná NoSQL\footnote{databázový koncept, ve kterém datové úložiště i zpracování dat používají jiné prostředky než tabulková schémata tradiční relační databáze} databáze
\item D3.js: framework pro práci s grafy
\item JQuery framework pro práci s GUI(Graphical user interface)
\end{itemize}