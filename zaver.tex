
\chapter[Závěr]{Závěr}

Vert.x je vysoce modifikovatelný webový framework založený na komunikaci v reálném čase napříč všemi částmi aplikace. Vysoká modularita a otevřenost platformy Vert.x přináší značné výhody. Již od počátku si kladl za cíl zjednodušit dosavadní možnosti vývoje a představit tak alternativu ke standardním nástrojům vývoje webových aplikací. Je to právě jednoduchost, univerzálnost a komplexnost řešení této platformy, které zláká nejednoho programátora aby minimálně zaexperimentoval s tímto nástrojem. V současné době se velmi progresivně rozšiřuje celý ekosystém okolo Vert.x novými nástroji a možnostmi. 

Práce představila unikátní filosofii a principy frameworku Vert.x a snažila se do jisté 
míry provést začínajícího uživatele hlavními aspekty vývoje s poukázáním na další související 
problematiku. Přitom se maximálně snažila omezit pouze na oblast vývoje webových aplikací, 
přestože někdy nebylo možné opomenout elementární souvislosti s vývojem distribuovaných webových aplikací. V práci byla představena platforma Vert.x jako nástroj pro distribuované webové aplikace. Byla popsána unikátní filozofie a terminologie této platformy.

V praktické části se podařilo vytvořit webovou aplikaci, která splňuje všechny aspekty moderní webové aplikace. Především pak komunikace v reálném čase bez náročných implementací či použití mnoha služeb a nástrojů. V aplikaci je možné jednoduchým a intuitivním způsobem přidávat, přejmenovávat a odebírat její jednotlivé body. Pokud má stejnou myšlenkovou mapu otevřeno více lidí, okamžitě vidí všechny změny, ostatních klientů. Aplikace používá volně šiřitelný software, který je ve většině případů špičkové úrovně. % Tento postup dovoluje, s relativně nízkými náklady, řešit velmi komplikované problémy. 
Možnosti pro vylepšení aplikace jsou jak na straně vizuální tak na straně funkcionální. Bylo by vhodné rozšířit aplikaci o možnost přihlášení a správy pouze svých myšlenkových či případné sdílení jednotlivých map s ostatními uživateli.

 
Z práce vyplývá, že se platforma Vert.x hodí pro vývoj webových aplikací, výhradně pak za použití s dalšími nástroji usnadňující vývoj MVC případně MVVM aplikací, například Spring frameworkem.

\section{Možnosti dalšího výzkumu}

Tak rozsáhlé téma jako jsou distribuované webové aplikace rozhodně nelze podrobně popsat v rámci 
jedné bakalářské práce. Na tuto práci proto mohou navazovat kolegové z fakulty či jiných 
vysokých škol. V závěru pro ně přináším dva zajímavá témata, na které již v této práci 
nezbyl prostor a rozhodně si zaslouží podrobnější analýzu.
 
\subsection{Distribuované výpočty}

\subsection{Srovnání}